%%%%%%%%%%%%%%%%%%%%%%%%%%%%%%%%%%%%%%%%%%%%%%%%%%%%%%%%%%%%
%			 	  Definciciones de Variables               %
%%%%%%%%%%%%%%%%%%%%%%%%%%%%%%%%%%%%%%%%%%%%%%%%%%%%%%%%%%%%
%%%%%%%%%%%%%%%%%%%%%
%     COLORES       %
%%%%%%%%%%%%%%%%%%%%%
\definecolor{R}{RGB}{176, 11, 11}
\definecolor{B}{RGB}{52, 75, 201}
\definecolor{G}{RGB}{20, 176, 18}
\definecolor{M}{RGB}{133, 71, 33}

%%%%%%%%%%%
%  TEXTO  %
%%%%%%%%%%%
\newtheorem{teo}{Teorema}[subsection]
\newtheorem{cor}{Corolario}[subsection]
\newtheorem{defi}{Definición}[subsection]
\newtheorem{obs}{Observación}[subsection]
\newtheorem{propo}{Proposición}[subsection]
\newtheorem{prop}{Propiedad}[subsection]
\newtheorem{ej}{Ejercicio}[subsection]

%%%%%%%%%%%%%%%%%%
%  MATEMATICAS   %
%%%%%%%%%%%%%%%%%%
% Este comando es para conjuntos numericos. Ej: \conj{R}
\newcommand{\conj}[1]{$\mathbb{#1}$ }
% Vectores
\newcommand{\vecAn}[1]{{$(a_1,a_2,\cdots,a_n )$ #1}}
\newcommand{\vecBn}[1]{{$(b_1,b_2,\cdots,b_n )$ #1}}
\newcommand{\vecdos}[2]{{(#1,#2)}}
\newcommand{\vectres}[3]{{(#1,#2,#3)}}
\newcommand{\dom}[1]{{\mathcal{D}}}
\newcommand{\origen}[1]{{$\mathcal{O}$}}
\newcommand{\modulo}[1]{{\vert{#1}\vert}}
\newcommand{\norma}[1]{{\Vert{#1}\Vert}}
\newcommand{\prodesc}[2]{{\langle #1,#2 \rangle}}
\newcommand{\cuerpo}[1]{\textbf{#1}}

\newcommand{\titulo}[1]{\subsection{\underline{\textbf{\color{B}{#1}}}}}
\newcommand{\ejercicio}[1]{\subsection{\textbf{\color{R}{#1}}}}
\newcommand{\solucion}[1]{\fbox{\textbf{Solución}}#1}
\newcommand{\resultado}[1]{\color{G}{#1}}